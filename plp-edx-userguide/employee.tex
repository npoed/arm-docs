\graphicspath{ {images/employee/} }
\section{Сотрудники}
Раздел \quotes{Сотрудники} содержит в себе различные функции по работе с сотрудниками. 
Перейти в раздел можно, выбрав его в верхнем меню (см. рис. ~\ref{img:employee:top_menu}).

\begin{figure}[H]
	\center{\includegraphics[width=1\linewidth]{top_menu}}
	\caption{Раздел \quotes{Сотрудники}}
	\label{img:employee:top_menu}
\end{figure}


\subsection{Роли и операции}
Раздел доступен пользователям роли \quotes{Администратор вуза}, в случае если вуз является поставщиком и потребителем. Доступные операции:
\begin{itemize}
	\item просмотр списка сотрудников;
	\item просмотр подробной информации о сотруднике;
	\item приглашение сотрудника на Платформу;
	\item приглашение сотрудника на Платформу списком;
	\item просмотр журнала действий сотрудников;
	\item назначение ролей.
\end{itemize}

\subsection{Список сотрудников}
Список сотрудников "--- это табличное представление всех пользователей, которым назначены какие"=либо роли в рамках 
данного вуза. Внешний вид списка представлен на рис.~\ref{img:employee:employee_list}.
Элементы управления табличными представления описаны в подразделе~\ref{sec:datatables}.

\begin{figure}[H]
	\center{\includegraphics[width=1\linewidth]{employee_list}}
	\caption{Список сотрудников}
	\label{img:employee:employee_list}
\end{figure}


Таблица содержит следующие столбцы:
\begin{itemize}
	\item ФИО сотрудника;
	\item адрес электронной почты;
	\item дата регистрации;
	\item флаг активности;
	\item последнее действие.
\end{itemize}

ФИО сотрудника является ссылкой на просмотр подробной информации о нем, описанной в подразделе~\ref{sec:employee_detail}.
В крайнем правом столбце таблицы находится кнопка перехода к форме редактирования ролей данного сотрудника в рамках 
данного вуза, описание которой находится в подразделе \ref{sec:university_role}.

Внешний вид диалога фильтрации списка сотрудников представлен на рис.~\ref{img:employee:employee_list_filter}.
Можно фильтровать список по следующим полям:

\begin{itemize}
	\item ФИО "--- текстовое поле;
	\item адрес электронной почты "--- текстовое поле;
	\item диапазон дат регистрации "--- виджеты выбора даты и времени 
	(описание виджета см. в подразделе~\ref{widget:date_time_picker});
	\item флаг активности "--- выпадающий список из вариантов да/нет.
\end{itemize}

\begin{figure}[H]
	\center{\includegraphics[height=6cm]{employee_list_filter}}
	\caption{Диалог фильтрации списка сотрудников}
	\label{img:employee:employee_list_filter}
\end{figure}


\subsection{Подробная информация о сотруднике} \label{sec:employee_detail}
Внешний вид страницы с подробной информацией о сотруднике представлен на рис.~\ref{img:employee:employee_detail}.
\begin{figure}[H]
	\center{\includegraphics[height=10cm]{employee_detail}}
	\caption{Подробная информация о сотруднике}
	\label{img:employee:employee_detail}
\end{figure}

Для просмотра доступны следующие поля:
\begin{itemize}
	\item дата регистрации;
	\item логин;
	\item адрес электронной почты;
	\item публичный email;
	\item телефон;
	\item дата рождения;
	\item регион;
	\item почтовый адрес;
	\item часовой пояс;
	\item образование;
	\item степень;
	\item университет;
	\item факультет;
	\item кафедра;
	\item список назначенных ролей;
	\item последние действия на Платформе.
\end{itemize}

Рядом со списком ролей пользователя находится кнопка перехода 
к форме редактирования ролей данного сотрудника в рамках данного вуза, 
описание которой находится в подразделе \ref{sec:university_role}.

\subsection{Приглашение сотрудника на Платформу}\label{sec:invite}

Внешний вид формы приглашения сотрудника на Платформу представлен на рис.~\ref{img:employee:invite}. 

\begin{figure}[H]
	\center{\includegraphics[width=1\linewidth]{invite}}
	\caption{Приглашение сотрудника на Платформу}
	\label{img:employee:invite}
\end{figure}

Для приглашения сотрудника на Платформу необходимо заполнить поля формы:
\begin{itemize}
	\item адрес электронной почты "--- текстовое поле;
	\item фамилия "--- текстовое поле;
	\item имя "--- текстовое поле;
	\item логин "--- текстовое поле.
\end{itemize}

Все поля формы являются обязательными для заполнения, если такое поле оставить не заполненным "--- появляется сообщение 
о необходимости его заполнения и блокируется кнопка сохранения изменений. 

При приглашении сотрудника на платформу осуществляется набор проверок:
\begin{itemize}
	\item адрес электронной почты должен быть уникален;
	\item электронная почта соответствует формату;
	\item имя пользователя должно быть уникально;
	\item имя пользователя должно содержать не менее 3 символов;
	\item имя пользователя может содержать только латинские символы, цифры и знак \_;
	\item фамилия, имя и отчество могут содержать только буквы, пробельные символы и дефис;
\end{itemize}

При нарушении этих проверок рядом с соответствующим полем формы появляются сообщения об ошибках прямо во время ввода.

После заполнения всех полей необходимо нажать на кнопку \quotes{Пригласить}.
В случае успеха осуществляется переход на страницу назначения прав приглашенному пользователю в рамках данного вуза 
(см. подраздел \ref{sec:university_role}), а также отображается сообщение (см. рис.~\ref{img:employee:invite_message}).

\begin{figure}[H]
	\center{\includegraphics[height=1.5cm]{invite_message}}
	\caption{Сообщение об успешном приглашении}
	\label{img:employee:invite_message}
\end{figure}


\subsection{Приглашение сотрудников на Платформу списком}
Внешний вид формы приглашения сотрудников на Платформу списком представлен на рис.~\ref{img:employee:mass_invite}. 

Для осуществления приглашения сотрудников списком необходимо нажать на кнопку \quotes{Обзор} и в появившемся диалоге 
выбрать CSV"=файл в универсальном формате, рассмотренном в п.~\ref{sec:csv_format}. 

\begin{figure}[H]
	\center{\includegraphics[width=1\linewidth]{mass_invite}}
	\caption{Приглашение сотрудников списком}
	\label{img:employee:mass_invite}
\end{figure}

После выбора файла необходимо нажать на кнопку \quotes{Загрузить}, после чего начнется загрузка файла.
При загрузке файла без записей отобразится ошибка (см. рис.~\ref{img:employee:mass_invite_empty_csv}).
\begin{figure}[H]
	\center{\includegraphics[height=3cm]{mass_invite_empty_csv}}
	\caption{Пустой CSV файл}
	\label{img:employee:mass_invite_empty_csv}
\end{figure}

При загрузке корректного файла создается фоновая задача для его обработки и появляется соответствующее сообщение 
(см. рис.~\ref{img:employee:mass_invite_task}), а также осуществляется переход на страницу списка сотрудников.

\begin{figure}[H]
	\center{\includegraphics[height=1.5cm]{mass_invite_task}}
	\caption{Сообщение о создании фоновой задачи}
	\label{img:employee:mass_invite_task}
\end{figure}
Для просмотра результатов выполнения задачи необходимо перейти в раздел \quotes{Фоновые задачи}, 
рассмотренный в п.~\ref{sec:mass_async_task}.

При приглашении сотрудников списком осуществляется набор проверок аналогичный рассмотренным в разделе 
приглашения сотрудника на Платформу (см. п.~\ref{sec:invite}).
При нарушении этих проверок соответствующие ошибки появятся в результирующей CSV со списком неуспешно 
приглашенных сотрудников.


\subsection{Журнал действий сотрудников}
Журнал действий сотрудников "--- это табличное представление истории действий сотрудников данного вуза на Платформе.
Внешний вид списка представлен на рис.~\ref{img:employee:log_list}. 
Элементы управления табличными представлениями описаны в подразделе~\ref{sec:datatables}.
\begin{figure}[H]
	\center{\includegraphics[width=1\linewidth]{log_list}}
	\caption{Журнал действий сотрудников}
	\label{img:employee:log_list}
\end{figure}

Таблица содержит следующие столбцы:
\begin{itemize}
	\item дата и время действия;
	\item сотрудник, совершивший действие;
	\item тип действия;
	\item текстовое описание действия.
\end{itemize}

ФИО сотрудника является ссылкой на просмотр подробной информации о нем, описанной в подразделе~\ref{sec:employee_detail}.

Внешний вид диалога фильтрации журнала действий представлен на рис.~\ref{img:employee:log_list_filter}.
Можно фильтровать список по следующим полям:

\begin{itemize}
	\item диапазон дат действия "--- виджеты выбора даты и времени 
	(описание виджета см. в подразделе~\ref{widget:date_time_picker});
	\item пользователь, совершивший действие "--- виджет выпадающего списка с автодополнением с возможностью множественного выбора 
	(описание виджета см. в подразделе~\ref{widget:autocomplete_with_multiselect});
	\item тип действия "--- виджет выпадающего списка с автодополнением с возможностью множественного выбора
	(описание виджета см. в подразделе~\ref{widget:autocomplete_with_multiselect}).
\end{itemize}

\begin{figure}[H]
	\center{\includegraphics[height=6cm]{log_list_filter}}
	\caption{Диалог фильтрации журнала действий}
	\label{img:employee:log_list_filter}
\end{figure}

В настоящий момент в журнал записываются следующие типы действий:

\begin{itemize}
	\item вход на Платформу;
    \item приглашение на Платформу;
    \item выдача полномочий;
    \item создание университета;
    \item редактирование университета;
    \item создание преподавателя;
    \item редактирование преподавателя;
    \item удаление преподавателя;
    \item создание курса;
    \item редактирование курса;
    \item удаление курса;
    \item создание сессии курса;
    \item перезапуск сессии курса;
    \item редактирование сессии курса;
    \item удаление сессии курса;
    \item зачисление студентов;
    \item отчисление студентов;
    \item изменение режима прохождения курса студентами;
    \item создание заявки на зачисление студентов;
    \item принятие заявки на зачисление студентов;
    \item отклонение заявки на зачисление студентов;
    \item удаление заявки на зачисление студентов;
    \item создание заявки на изменение режима прохождения сессии курса студентом;
    \item принятие заявки на изменение режима прохождения сессии курса студентом;
    \item отклонение заявки на изменение режима прохождения сессии курса студентом;
    \item удаление заявки на изменение режима прохождения сессии курса студентом;
    \item редактирование заявки;
    \item создание договора;
    \item редактирование договора;
    \item удаление договора;
    \item создание заявки на привязку студентов к вузу;
    \item принятие заявки на привязку студентов к вузу;
    \item отклонение заявки на привязку студентов к вузу.
\end{itemize}

Некоторые типы событий предполагают наличие в описании события вспомогательных ссылок для упрощения навигации. 
Так, при создании или редактировании университетов, преподавателей, курсов, сессий курсов, 
заявок на зачисление/изменение режима прохождения, договоров между вузом"=разработчиком и вузом"=потребителем,
при принятии или отклонении заявок содержится ссылка на созданную или измененную сущность.


Если тип действия связан с редактированием какой"=либо сущности, то в журнале отображается информация об изменениях, 
внесенных в результате редактирования. Просмотреть подробности изменения можно, 
нажав на кнопку \vcenteredinclude[height=20px]{plus} в крайнем правом столбце таблицы.
Внешний вид внесенных при редактировании изменений представлен на рисунке~\ref{img:employee:log_diff}.
\begin{figure}[H]
	\center{\includegraphics[width=1\linewidth]{log_diff}}
	\caption{Внесенные при редактировании изменения}
	\label{img:employee:log_diff}
\end{figure}

Для каждой измененной характеристики объекта отображается название характеристики, её старое и новое значение.

Для того, чтобы спрятать подробности по данному действию, нужно нажать на кнопку \vcenteredinclude[height=20px]{minus}.

Действия, связанные с выполнением фоновых задач, содержат ссылки на скачивания списка успешных и 
списка неуспешных результатов, а также ссылку на страницу подробной информации 
(см. рис.~\ref{img:employee:log_mass_async_task_list}).

\begin{figure}[H]
	\center{\includegraphics[width=1\linewidth]{log_mass_async_task_list}}
	\caption{Запись о выполнении фоновой задачи}
	\label{img:employee:log_mass_async_task_list}
\end{figure}

Страница подробной информации помимо полей, отображаемых в списке действий содержит детальную информацию 
об успешных и неуспешных результатах, представленную на рис.~\ref{img:employee:log_mass_async_task_detail}. 

\begin{figure}[H]
	\center{\includegraphics[width=1\linewidth]{log_mass_async_task_detail}}
	\caption{Детальная информация о выполнении фоновой задачи}
	\label{img:employee:log_mass_async_task_detail}
\end{figure}


\subsection{Назначение ролей в рамках вуза} \label{sec:university_role}

Назначение ролей в рамках вуза доступно пользователям роли \quotes{Администратору вуза} для вузов, которые являются поставщиком и потребителем, а также \quotes{Суперпользователю}. В случае, если у сотрудника
уже есть роли в рамках данного вуза, в форму редактирования прав можно попасть, нажав \vcenteredinclude[height=25px]{edit_btn} в 
таблице с сотрудниками. Если у сотрудника ещё нет ролей в рамках данного вуза, то назначить ему роли можно выбрав пункт 
\quotes{Назначение ролей} в разделе <<Сотрудники>>.

\subsubsection{Назначение ролей из таблицы сотрудников}

При нажатии на кнопку \vcenteredinclude[height=25px]{edit_btn} в таблице сотрудников пользователю будет показана форма, 
аналогичная показанной на рис.~\ref{img:employee:individual_form}. В данной форме будут добавлены уже имеющиеся у пользователя роли
с их параметрами. Все изменения будут сохранены только после нажатия кнопки <<Сохранить>> внизу формы.

\begin{figure}[H]
	\center{\includegraphics[width=1\linewidth]{individual_form}}
	\caption{Форма назначения ролей для действующего сотрудника}
	\label{img:employee:individual_form}
\end{figure}

Для добавления новой роли необходимо нажать на кнопку <<Добавить роль>> в нижней части формы. В результате появится часть формы для
добавления одной роли, предлагающая выбрать роль для назначения (рис.~\ref{img:employee:choose_role}). В зависимости от 
выбранной роли, на экране появится часть формы для выбора параметра роли. Возможны следующие типы параметров:
\begin{itemize}
	\item {\bf вуз}: неизменяемое поле, в котором выбран текущий вуз (рис.~\ref{img:employee:uni_role});
	\item {\bf курс}: для выбора используется неизменяемое поле <<вуз>> и выпадающий список с автодополнением <<курс>>. 
	Возможен выбор из курсов выбранного университета (рис.~\ref{img:employee:course_role});
	\item {\bf сессия курса}:  для выбора используется неизменяемое поле <<вуз>>, выпадающий список с автодополнением <<курс>>.
\end{itemize}

Описание работы с полями для выбора параметров см. в подразделе~\ref{widget:autocomplete}

\begin{figure}[H]
	\center{\includegraphics[width=1\linewidth]{individual_form_select}}
	\caption{Назначение новой роли пользователю: выбор роли}
	\label{img:employee:choose_role}
\end{figure}
\begin{figure}[H]
	\center{\includegraphics[width=1\linewidth]{uni_role}}
	\caption{Назначение новой роли пользователю: роль с параметром типа <<вуз>>}
	\label{img:employee:uni_role}
\end{figure}
\begin{figure}[H]
	\center{\includegraphics[width=1\linewidth]{course_role}}
	\caption{Назначение новой роли пользователю: Назначение новой роли пользователю: роль с параметром типа <<курс>>}
	\label{img:employee:course_role}
\end{figure}

Для удаления роли нужно нажать кнопку <<Удалить роль>> рядом с выбранной ролью.


\subsubsection{Назначение ролей для новых сотрудников}

При нажатии на пункт \quotes{Назначение ролей} в разделе \quotes{Сотрудники} пользователю показывается форма для выбора пользователя,
которому будут назначаться права (рис.~\ref{img:employee:choose_user_form}). После выбора сотрудника будут загружены имеющиеся роли (если они есть).
Далее работа с формой аналогична работе с формой, описанной в предыдущем разделе.

\begin{figure}[H]
	\center{\includegraphics[width=1\linewidth]{choose_user_form}}
	\caption{Форма назначения ролей для нового сотрудника}
	\label{img:employee:choose_user_form}
\end{figure}

\subsection{Назначение глобальных ролей} \label{sec:global_role}

Для \quotes{Суперпользователя} доступна глобальная форма для назначения ролей. Попасть в эту форму можно нажав на пункт <<Назначение ролей>> в
меню <<Мой профиль>> (рис.~\ref{img:employee:global_perms_menu}). Работа с данной формой аналогична работе с формой, описанной в 
подразделе~\ref{sec:university_role}, со следующими отличиями:
\begin{itemize}
	\item если для выбора параметра требуется выбор вуза, поле <<вуз>> является выпадающим списком с 
	автодополнением (см. подраздел~\ref{widget:autocomplete});
	\item если для выбора параметра требуется выбор курса, то сначала должен быть выбран вуз"=разработчик курса.
\end{itemize}

\begin{figure}[H]
	\center{\includegraphics[width=1\linewidth]{global_perms_menu}}
	\caption{Переход на глобальную форму назначения ролей}
	\label{img:employee:global_perms_menu}
\end{figure}

