\graphicspath{ {images/menu/} }
\section{Меню АРМ вуза}
Меню АРМ вуза представляет из себя набор разделов, в которых расположены пункты, позволяющие выполнять пользовательские операции. Внешний вид разделов представлен на рис.~\ref{img:menu:nav_menu}. Названия неактивных разделов отображаются синим цветом. Активный раздел - тот, в котором производится работа пользователя, выделен белым цветом.

\begin{figure}[H]
	\center{\includegraphics[width=1\linewidth]{nav_menu}}
	\caption{Разделы АРМ}
	\label{img:menu:nav_menu}
\end{figure}

Перечень разделов и операций приведён в следующем списке.

\begin{enumerate}

	\item Об университете\\
	Раздел доступен для пользователей любых вузов.
	\begin{itemize}
		\item просмотр и редактирование (кроме флага «Является только потребителем») информации об университете.
	\end{itemize}

	\item Преподаватели\\
	Раздел доступен только для пользователей вузов, которые являются разработчиками и потребителями. Доступные операции:
	\begin{itemize}
			\item просмотр списка преподавателей, создание, просмотр профиля;
			\item редактирование и удаление преподавателя.
	\end{itemize}

	\item Курсы\\
	Раздел доступен только для пользователей вузов, которые являются разработчиками и потребителями. Доступные операции:
	\begin{itemize}
			\item просмотр списка курсов, создание, просмотр карточки, редактирование и удаление всех курсов вуза;
			\item назначение авторов всех курсов вуза;
			\item просмотр списка сессий курсов, создание, просмотр карточки, редактирование, перезапуск и удаление всех сессий курсов вуза.
	\end{itemize}

	\item Сотрудники\\
	Раздел доступен только для пользователей вузов, которые являются разработчиками и потребителями. Доступные операции:
	\begin{itemize}
		\item просмотр списка сотрудников, профиля сотрудника; 
		\item приглашение сотрудников на Платформу;
		\item просмотр журнала действий сотрудников;
		\item назначение пользователям ролей на данный вуз и курсы вуза;
	\end{itemize}

	\item Студенты\\
	Раздел доступен для пользователей любых вузов. Операции, доступные для пользователей вузов, которые являются только потребителем:
	\begin{itemize}
		\item загрузка студентов списком;
		\item просмотр списка студентов вуза;
		\item просмотр подробной информации о студенте вуза;
		\item создание, редактирование, просмотр заявки на зачисление;
		\item создание, редактирование, просмотр заявки на изменение режима прохождения сессии курса;
	\end{itemize}
	Для пользователей вузов, которые являются поставщикам и потребителями помимо операций, перечисленных выше, доступны следующие:
	\begin{itemize}
		\item просмотр студентов, обучающихся на курсах вуза;
		\item зачисление студентов;
		\item отчисление студентов;
		\item изменение режима прохождения сессии курса студентами;
	\end{itemize}

	\item Аналитика\\
	Раздел доступен для пользователей любых вузов. Операции, доступные для пользователей вузов, которые являются только потребителем:
	\begin{itemize}
		\item просмотр аналитических сведений о прохождении курсов студентами своего вуза.
	\end{itemize}
	Для пользователей вузов, которые являются поставщикам и потребителями помимо операций, перечисленных выше, доступны следующие:
	\begin{itemize}
		\item просмотр аналитических сведений о прохождении курсов вуза.
	\end{itemize}

	\item Договоры\\
	Раздел доступен для пользователей любых вузов. Операции, доступные для пользователей вузов, которые являются только потребителем:
	\begin{itemize}
		\item просмотр списка и подробной информации о договорах на получение услуг.
	\end{itemize}
	Для пользователей вузов, которые являются поставщикам и потребителями помимо операций, перечисленных выше, доступны следующие:
	\begin{itemize}
		\item создание просмотр списка и подробной информации, редактирование договора на предоставление услуг;
		\item просмотр списка платежей;
		\item просмотр заявок на зачисление и управление статусом заявок;
		\item просмотр заявок на изменение режима прохождения сессий и управление статусом заявок.
	\end{itemize}

\end{enumerate}


