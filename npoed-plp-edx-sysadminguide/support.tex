\section{Плановые работы}
\begin{enumerate}
	\item Проверка логов на наличие ошибок
	
		Необходимо периодически проверять содержимое лог-файлов на наличие сообщений уровня \texttt{ERROR} и сообщать об их обнаружении.

	\item Ротация логов
	
		Если логирование происходит на диск, то необходимо позаботится о своевременной ротации лог-файлов
	
	\item Проверка очереди \texttt{RabbitMQ} на переполнение
	
		Необходимо периодически проверять очередь заданий \texttt{RabbitMQ}. Переполнение этой очереди свидетельствует о проблемах
		с обработкой заданий в \texttt{Celery}. В первую очередь нужно перезапустить демон \texttt{Celery}. Если это не помогло,
		то требуется анализ лог-файлов для обнаружения причины проблемы.
	
	\item Анализ логов медленных запросов \texttt{MySQL}
	
		Для обнаружения проблем с производительностью при работе с базой данных нужно включить в \texttt{MySQL} логирование медленных запросов.
		После этого необходимо периодически анализировать содержимое лог-файла с этими запросами.
\end{enumerate}

\section{Резервное копирование и восстановление}
Данные в \texttt{PLP} синхронизируются с другими подсистемами (\texttt{SSO}, \texttt{LMS, LCMS}), поэтому резервное копирование и восстановление
необходимо производить одновременно для всех подсистем, предварительно их остановив.

Резервное копирование \texttt{PLP} состоит в создании дампа базы данных \texttt{PLP} и копировании всех файлов проекта.

Восстановление, соответственно, предполагает восстановление проекта из копии, удаление базы данных и восстановление её из дампа.

